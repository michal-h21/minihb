\documentclass{article}
% \usepackage{fontspec}
\usepackage{luaotfload}
\usepackage{minihb}
\usepackage[english]{babel}

% we must turn off some features automatically added by luaotfload
\font\pokus={Amiri:script=arab;language=ara;-init,-medi,-fina} at 18pt

% but with Scheherazade these features doesn't matter
\font\arab={Scheherazade:language=ara;script=arab;} at 18pt

\font\noto={Noto Nastaliq Urdu:script=arab;language=urd;-init,-medi,-fina} at 18 pt

\font\libertine={Linux Libertine O:+onum;+frac} at 18pt
\begin{document}

\pokus
% \arab


\pardir TRT
% \textdir TRT


براغ (بالتشيكية: Praha, Česká republika، براها) هي عاصمة 

طالع أيضًا: إبراهيم بن يعقوب



\arab

طالع أيضًا: إبراهيم بن يعقوب

\libertine

Latin text in RTL paragraph. The region was settled grafika. VLTAVA %as early as the Paleolithic age.[14] Around the fifth

\noto 

پراگ (Prague) چیک جمہوریہ کا دارالحکومت اور


\pardir TLT
\arab
طالع أيضًا: إبراهيم بن يعقوب


% The region was settled as early as the Paleolithic age.[14] Around the fifth
% and fourth century BC, the Celts appeared in the area, later establishing
% settlements including an oppidum in Závist, a present-day suburb of Prague, and
% giving name to the region of Bohemia, "home of the Boii".[14][15] In the last
% century BC, the Celts were slowly driven away by Germanic tribes (Marcomanni,
% Quadi, Lombards and possibly the Suebi), leading some to place the seat of the
% Marcomanni king Maroboduus on the site of present-day Prague.[16][17] Around
% the area where present-day Prague stands, the 2nd century map of Ptolemaios
% mentioned a Germanic city called Casurgis.[18]

% \pardir TLT
% The region was settled %as early as the Paleolithic age.[14] Around the fifth
\end{document}

