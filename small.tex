\documentclass{article}
% \usepackage{fontspec}
\usepackage{luaotfload}
\usepackage{minihb}
\usepackage[czech]{babel}
\font\pokus={Amiri:script=arab;language=ara;} at 18pt
% \font\pokus={Amiri:language=ara;+ccmp;+rlig;+calt;+curs} at 18pt

% \font\pokus={Amiri:script=arab;language=ara;+fina;+medi;+init;+rlig;+ccmp} at 18pt
\font\arab={Scheherazade:language=ara;script=arab;} at 18pt

\font\libertine={Linux Libertine O:+onum;+frac} at 18pt
\begin{document}

\pokus


\pardir TRT
% \textdir TRT


% براغ (بالتشيكية: Praha, Česká republika، براها) هي عاصمة 

طالع أيضًا: إبراهيم بن يعقوب



\arab

طالع أيضًا: إبراهيم بن يعقوب

\libertine

The region was settled grafika %as early as the Paleolithic age.[14] Around the fifth

\pardir TLT
\arab
طالع أيضًا: إبراهيم بن يعقوب
% The region was settled as early as the Paleolithic age.[14] Around the fifth
% and fourth century BC, the Celts appeared in the area, later establishing
% settlements including an oppidum in Závist, a present-day suburb of Prague, and
% giving name to the region of Bohemia, "home of the Boii".[14][15] In the last
% century BC, the Celts were slowly driven away by Germanic tribes (Marcomanni,
% Quadi, Lombards and possibly the Suebi), leading some to place the seat of the
% Marcomanni king Maroboduus on the site of present-day Prague.[16][17] Around
% the area where present-day Prague stands, the 2nd century map of Ptolemaios
% mentioned a Germanic city called Casurgis.[18]

% \pardir TLT
% The region was settled %as early as the Paleolithic age.[14] Around the fifth
\end{document}

